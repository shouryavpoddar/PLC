\documentclass[12pt]{article}
\usepackage[margin=1in]{geometry}   % Adjust page margins as needed
\usepackage{amsmath,amssymb}        % For math symbols/environments
\usepackage{listings}               % For code listings
\usepackage[T1]{fontenc}            % Better font encoding
\usepackage[utf8]{inputenc}         % UTF-8 support
\usepackage{hyperref}               % Optional: hyperlinks in PDF

\title{Homework Solutions}
\author{Shourya Poddar}
\date{\today}

\begin{document}

    \maketitle

    \noindent
    \textbf{Note:} The following document is a direct LaTeX rendition of the Markdown content, including all math formulas, code blocks, and textual explanations.

    \section*{Question 1}

    \subsection*{Step 1: Configure Precedence (Separate into Nonterminals)}

    We split each precedence level into its own symbol, from lowest to highest:

    \begin{enumerate}
        \item \texttt{<A7>} -- handles assignment (\texttt{=})
        \item \texttt{<B7>} -- handles the ternary operator (\texttt{?:})
        \item \texttt{<C7>} -- handles logical OR (\texttt{||})
        \item \texttt{<D7>} -- handles logical AND (\texttt{\&&})
        \item \texttt{<E7>} -- handles logical NOT (\texttt{!})
        \item \texttt{<G7>} -- handles parentheses, booleans, and variables
        \item \texttt{<V7>} -- variable names
    \end{enumerate}

    \noindent\textbf{Roughly} (before enforcing associativity):
    \begin{verbatim}
<A7> → <V7> = <A7> | <B7>
<B7> → <C7>?<B7>:<B7> | <C7>
<C7> → <C7> || <D7> | <D7>
<D7> → <D7> && <E7> | <E7>
<E7> → !<E7> | <G7>
<G7> → ( <A7> ) | true | false | <V7>
<V7> → x | y | z
    \end{verbatim}

    \subsection*{Step 2: Add Right \& Left Associativities}

    \begin{description}
        \item[Right-associative:] \texttt{=}, \texttt{?:}, \texttt{!}
        \begin{itemize}
            \item The grammar \texttt{<A7> → <V7> = <A7> | <B7>} ensures we parse \texttt{=} from the \emph{right}.
            \item \texttt{<B7> → <C7>?<B7>:<B7> | <C7>} also parses chained ternaries from the \emph{right}.
            \item \texttt{<E7> → !<E7> | <G7>} is already right-associative because \texttt{!} recurses on \texttt{<E7>}.
        \end{itemize}

        \item[Left-associative:] \texttt{||}, \texttt{\&&}
        \begin{itemize}
            \item \texttt{<C7> → <C7> || <D7> | <D7>} means repeated \texttt{||} group left.
            \item \texttt{<D7> → <D7> && <E7> | <E7>} means repeated \texttt{\&&} group left.
        \end{itemize}
    \end{description}

    \noindent
    Putting it all together in the final unambiguous grammar:

    \begin{verbatim}
<A7> → <V7> = <A7>
     | <B7>

<B7> → <C7> ? <B7> : <B7>
     | <C7>

<C7> → <C7> || <D7>
     | <D7>

<D7> → <D7> && <E7>
     | <E7>

<E7> → ! <E7>
     | <G7>

<G7> → ( <A7> )
     | true
     | false
     | <V7>

<V7> → x | y | z
    \end{verbatim}

    \section*{Question 2}

    \subsection*{Characterizing the Attributes}
    \textbf{Inherited Attributes}
    \begin{itemize}
        \item Typetable: passed down to statements
        \item Initable: passed down to assignments and expressions
    \end{itemize}

    \textbf{Synthesized Attributes}
    \begin{itemize}
        \item type: computed up, property of expression
        \item typebinding: computed up when variable declaration
        \item initialized
    \end{itemize}

    \subsection*{1. Start Productions}

    \textbf{\texttt{<start1> → <stmt3> ; <start3>}}
    \begin{verbatim}
<stmt3>.typetable := <start1>.typetable
<stmt3>.inittable := <start1>.inittable
<start3>.typetable := <start1>.typetable  // possibly updated after <stmt3>
<start3>.inittable := <start1>.inittable  // possibly updated after <stmt3>
    \end{verbatim}

    \textbf{\texttt{<start2> → <stmt4>}}
    \begin{verbatim}
<stmt4>.typetable := <start2>.typetable
<stmt4>.inittable := <start2>.inittable
    \end{verbatim}

    \subsection*{2. Statement Productions}

    \textbf{\texttt{<stmt1> → <declare2>}}
    \begin{verbatim}
<declare2>.typetable := <stmt1>.typetable
<declare2>.inittable := <stmt1>.inittable
<stmt1>.typebinding := <declare2>.typebinding
    \end{verbatim}

    \textbf{\texttt{<stmt2> → <assign2>}}
    \begin{verbatim}
<assign2>.typetable := <stmt2>.typetable
<assign2>.inittable := <stmt2>.inittable
<stmt2>.initialized := <assign2>.initialized
    \end{verbatim}

    \subsection*{3. Declaration}

    \textbf{\texttt{<declare2> → <type3> <var>}}
    \begin{verbatim}
<type3>.typetable := <declare2>.typetable
<type3>.inittable := <declare2>.inittable
<declare2>.typebinding := ( <var>, <type3>.type )
<declare2>.initialized := ( <var>, false )
    \end{verbatim}

    \subsection*{4. Types}

    \textbf{\texttt{<type1> → int}}
    \begin{verbatim}
<type1>.type := int
    \end{verbatim}

    \textbf{\texttt{<type2> → double}}
    \begin{verbatim}
<type2>.type := double
    \end{verbatim}

    \subsection*{5. Assignment}

    \textbf{\texttt{<assign1> → <var> = <expression3>}}
    \begin{verbatim}
<expression3>.typetable := <assign1>.typetable
<expression3>.inittable := <assign1>.inittable
<assign1>.initialized := ( <var>, true )
    \end{verbatim}

    \subsection*{6. Expressions}

    \textbf{\texttt{<expression1> → <expression4> <op> <expression5>}}
    \begin{verbatim}
<expression4>.typetable := <expression1>.typetable
<expression5>.typetable := <expression1>.typetable
<expression4>.inittable := <expression1>.inittable
<expression5>.inittable := <expression1>.inittable
<expression1>.type := <expression4>.type
    \end{verbatim}

    \textbf{\texttt{<expression2> → <value4>}}
    \begin{verbatim}
<value4>.typetable := <expression2>.typetable
<value4>.inittable := <expression2>.inittable
<expression2>.type := <value4>.type
    \end{verbatim}

    \textbf{\texttt{<op> → + | - | * | ÷}}

    \subsection*{7. Values}

    \textbf{\texttt{<value1> → <var>}}
    \begin{verbatim}
<var>.typetable := <value1>.typetable
<var>.inittable := <value1>.inittable
<value1>.type := <var>.type
    \end{verbatim}

    \textbf{\texttt{<value2> → <integer>}}
    \begin{verbatim}
<value2>.type := integer
    \end{verbatim}

    \textbf{\texttt{<value3> → <float>}}
    \begin{verbatim}
<value3>.type := double
    \end{verbatim}

    \subsection*{8. Terminals}

    \begin{verbatim}
<var> → a legal name in the language
<integer> → a base 10 representation of an integer
<float> → a base 10 representation of a floating point number
    \end{verbatim}

    \section*{Question 3}

    \subsection*{(a) Type of Expression Must Match the Variable (Assignments)}

    \textbf{Where:} Any \texttt{<assign>} production, i.e. \texttt{<assign1> → <var> = <expression3>}.

    \textbf{Check:} Compare the variable’s declared type to the expression’s type. Formally:
    \begin{verbatim}
if <assign1>.typetable(<var>) != <expression3>.type then
   error("Type mismatch in assignment")
    \end{verbatim}

    \subsection*{(b) A Variable Must Be Declared Before Use}

    \textbf{Where:} Anywhere \texttt{<var>} appears
    \textbf{At} \texttt{<assign1> → <var> = <expression3>}:
    \begin{verbatim}
if <assign1>.typetable(<var>) == error then
   error("Use of undeclared variable in assignment")
    \end{verbatim}

    \textbf{At} \texttt{<value1> → <var>}:
    \begin{verbatim}
if <value1>.typetable(<var>) == error then
   error("Use of undeclared variable in expression")
    \end{verbatim}

    \subsection*{(c) A Variable Must Be Initialized Before Use}

    \textbf{Where:} Any place \texttt{<var>} is used but not assigned.
    \textbf{At} \texttt{<value1> → <var>}:
    \begin{verbatim}
if <value1>.inittable(<var>) == false then
   error("Use of uninitialized variable")
    \end{verbatim}

    \section*{Question 4}

    \subsection*{Initial Code}
    \begin{verbatim}
// Precondition: n ≥ 0 and A has n elements indexed from 0
1. bound = n;

/* Outer loop invariant:
   A[bound..(n-1)] is sorted (non-decreasing)
   AND
   every element in A[bound..(n-1)] ≥ all elements in A[0..bound-1]
*/
2. while (bound > 0) {

3.    t = 0;
4.    i = 0;

     /* Inner loop invariant (given):
        "A[t] is the largest element of A[0..t]"
     */

5.    while (i < bound - 1) {
6.       if (A[i] > A[i+1]) {
7.          swap = A[i];
8.          A[i] = A[i+1];
9.          A[i+1] = swap;
10.         t = i+1;
         }
11.      i++;
     }
12.   bound = t;
}

// Postcondition: A[0] ≤ A[1] ≤ ... ≤ A[n-1]
    \end{verbatim}

    \subsection*{A1. Outer Loop Invariant}

    \textbf{Claimed Invariant:}
    \emph{``A[bound..(n-1)] is sorted (non-decreasing), and every element in that segment is ≥ every element in A[0..(bound-1)].''}

    \textbf{1. Is It Correct?}
    Yes. This is a valid invariant for a bubble sort variant in which \texttt{bound} shrinks from \texttt{n} down to \texttt{0}.

    \textbf{2. Proof of Correctness}

    \begin{itemize}
        \item \textbf{Initialization:} At the very start, \texttt{bound = n}, so \texttt{A[n..(n-1)]} is an empty subarray (trivially sorted).
        \item \textbf{Maintenance:} Assume the invariant holds at the start of an iteration. The inner loop bubbles up the largest element to the boundary, and we set \texttt{bound = t}. The subarray remains sorted and ≥ the earlier elements.
        \item \textbf{Termination:} Eventually, \texttt{bound = 0} or \texttt{1}, meaning the whole array is sorted.
    \end{itemize}

    \subsection*{A2. Inner Loop Invariant}

    \textbf{Claimed Invariant:}
    \emph{``A[t] is the largest element of A[0..t].''}

    \textbf{1. Is It Correct?}
    Partly correct but needs refinement. The code sets \texttt{t} = \texttt{i+1} upon a swap, tracking the last swap position. If that position always holds the largest so far, then it is effectively correct.

    \textbf{2. Proof (Refined Statement)}

    \begin{itemize}
        \item \textbf{Initialization:} \texttt{t=0, i=0}; trivially \texttt{A[t]} is largest in \texttt{A[0..0]}.
        \item \textbf{Maintenance:} If a swap occurs, the bigger element ends up at \texttt{i+1}, so \texttt{t} moves there.
        \item \textbf{Termination:} Once the inner loop ends, \texttt{bound = t}, ensuring the sorted portion is extended.
    \end{itemize}

    \subsection*{B. Weakest Preconditions (WP) for Each Numbered Statement}

    \begin{enumerate}
        \item \textbf{WP(1):} \texttt{n ≥ 0} and \texttt{A} has \texttt{n} elements
        \item \textbf{WP(2):} Outer loop invariant holds
        \item \textbf{WP(3):} \texttt{bound > 0}; \texttt{t=0} doesn’t break it
        \item \textbf{WP(4):} Outer loop invariant; now \texttt{i=0}, \texttt{t=0}
        \item \textbf{WP(5):} Inner loop invariant established, \texttt{i=0}, \texttt{t=0}
        \item \textbf{WP(6):} \texttt{i < bound - 1}, valid indexing
        \item \textbf{WP(7,8,9):} Swap is safe, \texttt{A[t]} remains largest in \texttt{A[0..t]}
        \item \textbf{WP(10):} \texttt{t = i+1} if a swap occurred, preserving the largest element property
        \item \textbf{WP(11):} \texttt{i++} still keeps the invariant
        \item \textbf{WP(12):} \texttt{bound = t}, re-establish outer loop invariant
    \end{enumerate}

    \subsection*{C. Is WP(1) Inferred by the Code’s Precondition?}

    Yes, the code’s given precondition is \texttt{n ≥ 0 and A has n elements}, which matches WP(1).

    \section*{Question 5}

    We want to define a function

    $$
    M_{\text{state}}\bigl(\langle \text{syntax} \rangle, \sigma\bigr)
    $$

    that takes a \emph{syntax tree} (for an \texttt{<assign>}, \texttt{<if>}, or \texttt{<while>}) and an \emph{initial state}
    $$\sigma$$,
    and returns a \emph{new state}.

    \subsection*{Assumptions and Helper Mappings}

    \begin{itemize}
        \item \(\displaystyle M_{\text{int}}(\langle \text{expression} \rangle, \sigma)\): Evaluates a numeric expression in state \(\sigma\).
        \item \(\displaystyle M_{\text{boolean}}(\langle \text{condition} \rangle, \sigma)\): Evaluates a condition in state \(\sigma\), returning `true`/`false`.
        \item \(\displaystyle M_{\text{name}}(\langle \text{var}\rangle)\): Extracts the string name of the variable (no state needed).
        \item \(\text{AddBinding}(\text{name}, \text{value}, \sigma)\): Returns a new state with `name → value`.
        \item \(\text{RemoveBinding}(\text{name}, \sigma)\): Returns a new state removing that name’s old binding.
    \end{itemize}

    \subsection*{Side Effects}

    Expressions and conditions can change the values of variables. Hence, \(M_{\text{int}}\) and \(M_{\text{boolean}}\) might return a pair \((\text{value}, \sigma')\).

    \subsection*{A. Assignment}

    \textbf{Syntax Rule:}
    \begin{verbatim}
<assign> → <var> = <expression>
    \end{verbatim}

    We first evaluate \(\langle \text{expression}\rangle\) in \(\sigma\), obtaining \((v,\sigma_1)\). Then we add a new binding of \(\langle \text{var}\rangle\) to \(v\) in \(\sigma_1\).

    \[
        M_{\text{state}}\bigl(\langle \text{var}\rangle = \langle \text{expression}\rangle,\;\sigma\bigr)
        = \text{let } (v,\sigma_1) = M_{\text{int}}\bigl(\langle \text{expression}\rangle,\;\sigma\bigr)
        \text{ in }
        \text{AddBinding}\Bigl(
        M_{\text{name}}\bigl(\langle \text{var}\rangle\bigr),
        v,\;\sigma_1
        \Bigr).
    \]

    \subsection*{B. If-Then-Else}

    \textbf{Syntax Rule:}
    \begin{verbatim}
<if> → if <condition> then <statement1> else <statement2>
    \end{verbatim}

    \[
        \begin{aligned}
            M_{\text{state}}\bigl(\text{if } \langle \text{condition}\rangle \text{ then } S_1 \text{ else } S_2,\;\sigma\bigr)
            &= \text{let } (b,\sigma_1) = M_{\text{boolean}}\bigl(\langle \text{condition}\rangle,\;\sigma\bigr) \\
            &\quad \text{in }
            \begin{cases}
                M_{\text{state}}(S_1,\;\sigma_1), & \text{if } b = \text{true},\\
                M_{\text{state}}(S_2,\;\sigma_1), & \text{if } b = \text{false}.
            \end{cases}
        \end{aligned}
    \]

    \subsection*{C. While Loop}

    \textbf{Syntax Rule:}
    \begin{verbatim}
<while> → while <condition> <loop body>
    \end{verbatim}

    We repeatedly evaluate the condition; if true, we run the loop body and repeat; else we exit.

    \[
        \begin{aligned}
            M_{\text{state}}\bigl(\text{while } \langle \text{condition}\rangle \; S,\;\sigma\bigr)
            &= \text{let } (b,\sigma_1)
            = M_{\text{boolean}}\bigl(\langle \text{condition}\rangle,\;\sigma\bigr) \\
            &\quad \text{in }
            \begin{cases}
                M_{\text{state}}\Bigl(\text{while } \langle \text{condition}\rangle \; S,\;\sigma_2\Bigr),
                & \text{if } b = \text{true},\\
                \sigma_1,
                & \text{if } b = \text{false}.
            \end{cases}\\
            &\quad \text{where } \sigma_2 = M_{\text{state}}(S,\;\sigma_1).
        \end{aligned}
    \]

    \textbf{Step-by-step meaning}:
    \begin{enumerate}
        \item Evaluate \(\langle \text{condition}\rangle\) in \(\sigma\) to get \((b,\sigma_1)\).
        \item If \(b = \text{true}\), evaluate \(S\) in \(\sigma_1\), producing \(\sigma_2\). Then do the `while` again with \(\sigma_2\).
        \item If \(b = \text{false}\), terminate and return \(\sigma_1\).
    \end{enumerate}

\end{document}
